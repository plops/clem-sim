\documentclass[11pt,abstracton,titlepage]{scrartcl}
\usepackage{graphicx}
\usepackage{hyperref}
\usepackage{units}
\usepackage{amsmath}
\usepackage{verbatim}
\usepackage{listings}
\usepackage{color}
\usepackage{textcomp}
\definecolor{listinggray}{gray}{0.9}
\definecolor{lbcolor}{rgb}{0.9,0.9,0.9}
\lstset{
	backgroundcolor=\color{lbcolor},
	tabsize=4,
	rulecolor=,
	language=matlab,
        basicstyle=\scriptsize,
        upquote=true,
        aboveskip={1.5\baselineskip},
        columns=fixed,
        showstringspaces=false,
        extendedchars=true,
        breaklines=true,
        prebreak = \raisebox{0ex}[0ex][0ex]{\ensuremath{\hookleftarrow}},
        frame=single,
        showtabs=false,
        showspaces=false,
        showstringspaces=false,
        identifierstyle=\ttfamily,
        keywordstyle=\color[rgb]{0,0,1},
        commentstyle=\color[rgb]{0.133,0.545,0.133},
        stringstyle=\color[rgb]{0.627,0.126,0.941},
}

\newcommand{\abs}[1]{\left| #1 \right|}
\newcommand{\avg}[1]{\langle #1 \rangle}
\newcommand{\figref}[1]{Figure~\ref{#1}}

\title{CLEM PAM}
\author{Martin Kielhorn}

\begin{document}
\section{State of the Art}
If a fluorescent plane is imaged with a fluorescence widefield
microscope, the intensity of the image is constant no matter if the
plane is in-focus or not. 
\subsection{Apotome}
Using non-uniform excitation light the Wilson group was able to
circumvent this problem (\cite{1997Neil}). They project a grid pattern
into the sample. For a fluorescent plane sample the modulation in the
image is highest, when the sample is in-focus. If the plane is moved
out of focus, the modulation decreases. Hence it is possible to achieve
z-sectioning in a widefield microscope.

For one section $I_p$ they combine three grating images $I_i$ (grating
period $1/\nu$) with different phases:
\begin{align}
  I_p=\abs{I_1+I_2e^{i2\pi/3}+I_3e^{i4\pi/3}}
  \sim\abs{ 2 \frac{J_1(2u\nu(1-\nu/2))}{u\nu(1-\nu/2)}}.
\end{align}

The last term is their result for the intensity of a fluorescent plane
with the defocus $u$. If no grating is displayed ($\nu=0$) the
resulting image $I_p$ is constant and the microscope has no sectioning
capability. A fine grating gives best sectioning capability (but one
has to make a tradeoff as a fine grating will give very low contrast
in a thick specimen).

For this method is necessary to capture three images to generate one
sectioned slice. This can be a problem if the movement of the grating
isn't perfect or if the object moves during these exposures.

\subsection{HiLo}
This problem is addressed by Jerome Mertz et al.\ (\cite{2008Lim},
\cite{2009Santos}). They realised that only two images need to be
captured for each z-sectioned slice.
An image with uniform illumination contains contributions
of both in-focus and out-of-focus fluorophores:
\begin{align}
  I_u(x,y)=I_\textrm{in}(x,y)+I_\textrm{out}(x,y).
\end{align}
Due to the structure of the OTF out-of-focus light is limited to low
spatial frequencies.

The modulated part of the non-uniformly illuminated image contains only
information of the in-focus object structure:
\begin{align}
  I_n(x,y)&=I_\textrm{in}(x,y)(1+M
  \sin(\kappa x+\varphi))+I_\textrm{out}(x,y),\\
  \kappa&=\frac{2\pi}{\lambda}n\sin(\alpha)\nu.
\end{align}

The sectioned image is a combination of high-frequency components of
the widefield image $I_u$ and the low-frequency components of the
modulated part of the non-uniform image $I_n$.

\subsubsection{Grating}

\subsubsection{Speckle}
In the older publication (\cite{2008Lim}) a random speckle pattern is
projected into the sample (presumably because this is experimentally
easier than projecting a grating). For completeness I describe my
reimplementation of their method. However I only test it on
numerically simulated data with a projected grating.

In order to process the non-uniform image it is divided into
subregions.  As a measure of local contrast in each of these
subregions the relative standard deviation is calculated 
(the operator $\avg{\cdot}$ indicates averaging):
\begin{align}
  c_n=\frac{\sqrt{\avg{I_n^2}-\avg{I_n}^2}}{\avg{I_n}}.
\end{align}
The relative standard deviation $c_n$ will be zero for image regions
without in-focus contributions. Its value increases as the modulation
of the non-uniform illumination increases (that's what we
want). However $c_n$ also increases if there is a small in-focus
feature. In that we are not interested as we only want to build up an
image of in-focus fluorophores with low spatial frequencies. To
correct for that we refer to the most complicated part of the paper
(in my opinion) -- its equation (4). I will repeat their argument
here.

Let $O$ be the image intensity obtained with perfectly uniform
illumination and $S$ the image intensity for the non-uniform
illumination a uniform object. Then the uniformly illuminated image of
the object is:
%FIXME Obviously I still don't understand it!
\begin{align}
  I_u\approx (\avg{O}+\delta O)\avg{S}.
\end{align}
Here the change of $I_u$ due to some object variations $\delta O$ can
only come from in-focus contributions.

\begin{align}
\label{in}
  I_n\approx (\avg{O}+\delta O)(\avg{S}+\delta S).
\end{align}
Here too, $\delta O$ and $\delta S$ come from the focus plane.

Now by squaring equation \ref{in}
\begin{align}
  I_n^2\approx \avg{O}\avg{S} + \delta O \avg{S} +\delta S\avg{S} +
  \delta S \delta O.
\end{align}

blablabla...

Somehow it turns out that the corrected relative standard deviantion
of the non-uniform illumination can be calculated like this:
\begin{align} 
  c_s^2=\frac{c_n^2-c_0^2}{1+c_0^2}.
\end{align}
With the relative standard deviation $c_0$ of the widefield image
$I_u$.


\section{On the numerical test sample}
\begin{figure}[htb]
  \centering
  \includegraphics{grating_xz}

  \caption{Left: x-z-Section %\verb!imgrat(:,64,:)! of the coherent
    image of the grating in object space.}
  \label{fig:grating}
\end{figure}

As a test object I create a sphere shell, a line and a rectangle.
That is shown in the following code.  The objective is a NA1.4 oil
with coherent \unit[473]{nm} excitation light and detection is at
\unit[520]{nm} (incoherent).

Some images of this simulation are shown in \figref{fig:grating}.

\begin{lstlisting}%%
n=128;
nmperpixel=100;
sz=2;
znmperpixel=sz*nmperpixel;
%% vector psf for excitation illumination
asf=squeeze(kSimPSF({'lambdaEx',473;'na',NA;'ri',1.518;...
    'sX',n;'sY',n;'sZ',sz*n;...
    'scaleX',nmperpixel;'scaleY',nmperpixel;'scaleZ',znmperpixel;...
    'computeASF',1}));


%% grating of period P
% one pixel is 100nm so the period is P*100nm
P=6;
grat2d=(mod(xx(n,n),P)>(floor(P/2)-1))*1000;
%% fill a 3d grating
grat=newim(n,n,sz*n);
grat(:,:,(sz*n)/2)=grat2d(:,:);
kgrat=ft(grat);

%% coherent imaging
imgratx=ift(kgrat.*ft(squeeze(asf(:,:,:,0))));
imgraty=ift(kgrat.*ft(squeeze(asf(:,:,:,1))));
imgratz=ift(kgrat.*ft(squeeze(asf(:,:,:,2))));
imgrat=abs(imgratx).^2+abs(imgraty).^2+abs(imgratz).^2

%% for imaging the fluorescence light
psf=kSimPSF({'lambdaEx',473;'lambdaEm',520;'na',1.4;'ri',1.518;...
        'sX',n;'sY',n;'sZ',sz*n;...
        'scaleX',nmperpixel;'scaleY',nmperpixel;'scaleZ',znmperpixel});
%% ft thereof
kpsf=ft(psf);

%% as an object I want a hollow sphere
% I define it in k-space
kobj=sinc(rr(kpsf)./2).*sinc(rr(kpsf)./0.7);

% this is the sphere
obj=rr(psf).*abs(ift(kobj));
maximum=max(obj);
% in-focus rectangle in right top
obj(83:114,23:43,(sz*n)/2)=4*maximum;
% in-focus line on the left
obj(21:21,40:90,(sz*n)/2)=12*maximum;
\end{lstlisting}
\bibliographystyle{plain}
\bibliography{clempam}
\end{document}
